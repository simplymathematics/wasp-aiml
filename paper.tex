\documentclass{article}
\usepackage[utf8]{inputenc}

\title{WASP: AIML Module 2}
\date{June 2022}

\begin{document}

\maketitle

\section{Logistic Regression}
Logistic regression uses an s-shaped function that maps a set up inputs to a value between 0 and 1, defined by:

$$
p(y = 1| t) =  \frac{e^z}{1 + e^z}
$$

where $z$ is a function, of some weights, inputs and biases ($ w, x, b$, respectively), such that

$$
z(x) = w \cdot x + b
$$

and 

$$
y(x) = 0 \textrm{ if } z(x) < .5, \textrm{ else } 1
$$


The loss can be defined as:

$$
L(x) = \frac{1}{n} \sum_{i = 1}^n log \bigg( 1 + e ^{-y_i(x)} \bigg)
$$

where n is the number of samples. Our objective is to minimize the total loss across all of the samples by changing the parameters $w, b$. Since this function is `nice' in the Calculus sense, we can minimize it analytically, using the gradient.

\subsection{Gradient}

To generalize this further, for any $z(x) = \theta \cdot x$

$$
log(p(x)) = log \bigg( \frac{1}{1+e^{-\theta x}} \bigg)= -log \bigg( 1 + e^{^-\theta x} \bigg)
$$

which yields

$$
L(x) = \frac{1}{n} \sum_{i = 1}^n -log \bigg( y_i \cdot \theta \cdot  x_i -log(1 + e ^-\theta x)  \bigg).
$$

Therefore

$$
\frac{\partial}{\partial \theta_j} y_i \cdot \theta \cdot x_{i,j} = y_i \cdot x_{i,j}
$$

and 

$$
\frac{\partial}{\partial \theta_j} -log(1 + e ^-\theta x)  = x_{j,i}z(x_i)
$$

such that

$$
\frac{\partial}{\partial \theta_j} = L(\theta) \frac{1}{n} \sum_{i = 1}^m(z(x^i) - y^i) \cdot x_{i,j}
$$

This is verified in the attached jupyter notebook.

\section{Experiments}
I looked at both the feature mapping experiments as well as the learning rate experiments. 
\subsection{Feature Mapping}
In this experiment, we used the original feature map to derive two more based on power transformations of the original data using powers 2 and 3 respectively.
\subsection{Learning Rate}
\subsubsection{Stochastic Gradient Descent}
\subsubsection{Accelerated Learning}
\subsubsection{Adaptive Learning}
\end{document}
